% Mengubah keterangan `Abstract` ke bahasa indonesia.
% Hapus bagian ini untuk mengembalikan ke format awal.
\renewcommand\abstractname{Abstrak}

\begin{abstract}

  % Ubah paragraf berikut sesuai dengan abstrak dari penelitian.
  Teknologi informasi yang ada saat ini berkembang pesat, dan dapat menawarkan peningkatan efisiensi di berbagai bidang. Dalam lingkup smart building, peningkatan efisiensi yang memungkinkan diantaranya kemudahan manajemen, penghematan biaya, peningkatan kelestarian lingkungan, dan lainnya. Penelitian ini bertujuan untuk mengetahui cara mendeteksi okupansi, dan mengembangkan alat yang dapat mendeteksi ketersediaan ruang berdasarkan okupansi menggunakan kamera. Sistem ini akan dilengkapi dengan teknologi deep learning, sehingga kamera dapat mendeteksi ada tidaknya orang dalam ruangan. Sistem ini diharapkan dapat dipasang di gedung perkuliahan atau kantor dosen. Sistem ini berpotensi untuk meningkatkan optimalisasi penggunaan ruang yang efektif, penghematan biaya, peningkatan kelestarian lingkungan, dan lainnya.

\end{abstract}

% Mengubah keterangan `Index terms` ke bahasa indonesia.
% Hapus bagian ini untuk mengembalikan ke format awal.
\renewcommand\IEEEkeywordsname{Kata kunci}

\begin{IEEEkeywords}

  % Ubah kata-kata berikut sesuai dengan kata kunci dari penelitian.
  \textit{Internet of Things}, Okupansi, \textit{Smart Building}

\end{IEEEkeywords}

% Ubah judul dan label berikut sesuai dengan yang diinginkan.
\section{Pendahuluan}
\label{sec:pendahuluan}

% Ubah paragraf-paragraf pada bagian ini sesuai dengan yang diinginkan.

Teknologi informasi yang ada saat ini berkembang pesat, dan dapat menawarkan peningkatan efisiensi di berbagai bidang. Dalam lingkup smart building, peningkatan efisiensi yang memungkinkan diantaranya kemudahan manajemen, penghematan biaya, peningkatan kelestarian lingkungan, dan lainnya. Internet of things (IoT) adalah integrasi dari beberapa teknologi untuk menyediakan layanan cerdas (smart services) di lingkungan yang cerdas. 
IoT ini sangat mungkin untuk diterapkan di kampus untuk membantu kegiatan operasional.


Penerapan IoT di lingkungan kampus ini berkaitan dengan konsep ‘kampus cerdas’, dan ini dapat meningkatkan pengalaman belajar mahasiswa, keamanan kampus dan juga efisiensi operasional [1]. 
Salah satu penerapan yang memungkinkan dari sistem tersebut di universitas adalah untuk mendukung pengguna untuk menemukan ruang yang tersedia. Dengan sistem manajemen ruang yang optimal, penggunaan ruangan ini dilakukan dengan efisien dengan tujuan memaksimalkan penggunaan ruang sekaligus meminimalkan biaya operasi dan pemeliharaan. Optimalisasi penggunaan ruang ini juga mengarah langsung pada penghematan energi. Lebih lanjut lagi, informasi mengenai okupansi ini berguna untuk perencanaan strategis antara lain untuk mengidentifikasi permintaan untuk jenis ruang tertentu [2]. Ini dapat menjadi salah satu upaya untuk memaksimalkan pemanfaatan ruang kuliah yang efektif dengan modifikasi manajemen operasi yang ada. Di sisi lain, manajemen ruang yang optimal memungkinkan untuk menutup beberapa ruang kuliah sehingga mengurangi biaya sewa untuk universitas atau mengizinkan ruang kuliah untuk disewakan pihak luar untuk menghasilkan pendapatan tambahan bagi bangunan kampus yang ada [3]. 


Berdasarkan latar belakang yang telah dipaparkan, rumusan masalah yang akan diselesaikan adalah bagaimana cara untuk mendeteksi okupansi, dan
bagaimana cara mengembangkan alat yang dapat mendeteksi ketersediaan ruang secara otomatis berdasarkan okupansi menggunakan kamera.


Ruang lingkup penelitian ini adalah penerapan IoT pada lingkungan kampus untuk mewujudkan {\it smart building}. Sistem yang dikembangkan menggunakan mikrokontroler, dan akan memggunakan kamera sebagai sensor untuk mendapatkan data citra.
Software dikembangkan hanya untuk sistem perhitungan, dan hanya sampai ke pengiriman hasil perhitungan.


Tujuan dari penelitian ini ini adalah untuk mengetahui cara mendeteksi okupansi, dan mengembangkan alat yang dapat mendeteksi ketersediaan ruang secara otomatis berdasarkan okupansi menggunakan kamera. Produk jadi hasil perkembangan dibuat menggunakan komponen yang ekonomis, memiliki ukuran keseluruhan yang ringkas, dan  biaya keseluruhan yang ekonomis


Pembahasan pada paper ini dimulai dengan presentasi mengenai penelitian lain (Bagian \ref{sec:penelitianterkait}).
Kemudian dilanjutkan dengan penjelasan mengenai arsitektur dari sistem yang dibuat (Bagian \ref{sec:arsitektur}).
Berdasarkan hal tersebut, kami menunjukkan lorem ipsum (Bagian \ref{sec:loremipsum}).
Terakhir, didapatkan kesimpulan dari penelitian yang telah dilakukan (Bagian \ref{sec:kesimpulan}).

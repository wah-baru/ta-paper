% Ubah judul dan label berikut sesuai dengan yang diinginkan.
\section{Penelitian Terkait}
\label{sec:penelitianterkait}

% Ubah paragraf-paragraf pada bagian ini sesuai dengan yang diinginkan.

Makalah "\emph{Experimental Evaluation of Internet of Things in the Educational Environment}" oleh Amr Elsaadany dan Mohamed Soliman mempelajari potensi manfaat dan dampak Internet of Things (IoT) di lingkungan pendidikan. Penulis berpendapat bahwa IoT memiliki potensi untuk merevolusi pendidikan dengan memberikan siswa pengalaman belajar yang dipersonalisasi, meningkatkan kolaborasi dan komunikasi, dan membuat pembelajaran lebih menarik dan interaktif.
Penulis melakukan evaluasi eksperimental IoT di lingkungan universitas. Mereka menggunakan berbagai perangkat IoT, termasuk sensor, aktuator, dan tag RFID, untuk mengumpulkan data tentang perilaku dan kinerja siswa. Data tersebut digunakan untuk mengembangkan model pembelajaran yang dipersonalisasi untuk setiap siswa. Model ini kemudian digunakan untuk memberi siswa umpan balik dan saran yang disesuaikan untuk perbaikan.

Rahman et al. (2020) dalam makalahnya menyajikan model pemanfaatan ruang untuk institusi pendidikan tinggi (PT). Model dikembangkan dengan wawancara dan diskusi kelompok terarah dengan pemangku kepentingan PT. Model ini didasarkan pada model system development lifecycle (SDLC) dan menggunakan diagram alir data untuk membuat prototipe model.
Model ini terdiri dari empat komponen utama:
\begin{itemize}
    \item Input: Komponen ini meliputi data fisik ruangan PT, seperti jumlah ruangan, ukuran ruangan, dan tipe ruangan.
    \item Proses: Komponen ini mencakup proses pengumpulan data pemanfaatan ruang, analisis data, dan pengembangan rekomendasi untuk peningkatan pemanfaatan ruang.
    \item Keluaran: Komponen ini mencakup laporan pemanfaatan ruang dan rekomendasi untuk meningkatkan pemanfaatan ruang.
    \item Umpan balik: Komponen ini mencakup umpan balik dari pemangku kepentingan LPT tentang model dan rekomendasinya.
\end{itemize}
Model tersebut diuji di salah satu perguruan tiinggi di Malaysia dan hasilnya menunjukkan bahwa model tersebut mampu meningkatkan pemanfaatan ruang. Model tersebut adalah alat yang berharga untuk HEI yang ingin meningkatkan pemanfaatan ruang dan mengurangi biaya.

Saffari et al. (2021) mengusulkan sistem deteksi okupansi kamera bebas baterai yang menggunakan kamera beresolusi rendah, modul komunikasi backscatter, dan Raspberry Pi 4 Model B. Kamera memanen energi dari cahaya sekitar dan mentransmisikan data ke Raspberry Pi menggunakan komunikasi backscatter. Raspberry Pi menjalankan model pembelajaran mendalam untuk mendeteksi keberadaan manusia.


% Contoh pembuatan persamaan ilmiah.
\begin{equation}
  \label{eq:hukumpertama}
  \sum \mathbf{F} = 0\; \Leftrightarrow\; \frac{\mathrm{d} \mathbf{v} }{\mathrm{d}t} = 0.
\end{equation}

\lipsum[6-7]
